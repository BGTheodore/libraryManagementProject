\chapter{Organisation}
        \section{Approche de travail}
\paragraph{}
Afin de réaliser au mieux ce travail, nous avons quelque peu respecté la methode agile,
nous permettanat alors de revenir en arrière en cas de souci. Il est vrai que nous
n'avons pas gardé un rapport étroit avec notre client. Néanmoins, nous avons respecté 
les autres contraintes liées à la méthodologie indiquée.
        \section{Methodologie}
\paragraph{}
Le cahier des charges ayant été imposé par le client, nous avons utilisé la méthodologie
Scrum afin de respecter au mieux les désirs de ce dernier. Néanmoins, nous avons dû prendre
certaines décisions de par nos points de vue. Ensemble, l'équipe s'est mis d'accord sur un
cahier des charges spécifiant des caractéristiques plus contraignantes afin que le travail
reste uniforme à tous les points lorsque chacun travail sur son module. \par 
D'un autre côté, un délai a dû être imposé pour la réalisation de chaque tâche spécifique
pour éviter que ce projet continue de façon illimitée. À la fin de chaque délai, l'équipe
se rencontre pour mettre en commun les différents travaux, vérifier que la fin correspond
parfaitement aux attentes de départ. En fin de compte, de nouvelles dispositions sont 
prises en prenant en compte les résultats escompté et obtenu.
        \section{Répartition des tâches}
\paragraph{}
Au cours de la première étape, à savoir l'analyse, chacun a eu à faire des recherches
sur l'objectif général du projet. Les trois membres ont dû apporter leur contribution
en faisant des recherches sur le pourquoi et le comment de ce projet. L'analse de
l'existant a été un plus crucial avant de nous lancer dans la conception. \par 
Pour rester efficient au cours de la conception, chacun des membres a eu un temps précis
pour réaliser un ensemble de diagramme précis, avec les explications nécessaires. Au final,
nous avons mis en commun les différents travaux, corrigé ce qu'il fallait et complété ce
qui semblait manquer. \par 
En fin de compte, pour la réalisation, chacun a eu son module à générer. Mais vu que 
chacun de nous avait son niveau en terme de programmation, le travail de l'un a du 
empiéter sur celui de l'autre de façon continue pour que chacun en apprenne le 
maximum.