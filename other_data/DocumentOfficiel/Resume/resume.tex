\section*{Résumé}
\paragraph{}
Afin de numériser la gestion des prêts dans une bibliothèque 
universitaire, cette dernière souhaite mettre en place une 
application capable de contrôler les différentes actions
 pouvant avoir lieu au sein de ladite bibliothèque. 
 De façon générale, elle est gérée par un chargé des 
 inscriptions et des relances des lecteurs quand ceux-ci 
 n’ont pas rendu leurs ouvrages au-delà du délai autorisé. 
 D’un autre côté, les bibliothécaires sont chargés de gérer 
 les emprunts et la restitution des ouvrages ainsi que 
 l’acquisition de nouveaux ouvrages. \par
Afin d’assurer l’intégrité de cette gestion, l’application 
devra définir les différentes tâches propres à chaque 
utilisateur et respecter les règlements internes de la 
bibliothèque. \par

\paragraph{Utilisateurs directes de l’application}
\par 
L’application sera continuellement en possession de deux 
utilisateurs spécifiques: un administrateur et un 
bibliothécaire. Mais, pour décongestionner l’espace, 
une interface sera accessible par les inscrits afin de 
consulter, à distance, un catalogue. Ainsi, chaque utilisateur aura des droits prédéfinis par 
l’application. Donc, pour avoir accès à l’application, 
ils devront d’abord s’authentifier. \par

\begin{enumerate}
    \item Gestionnaire
\\ Deux types de gestionnaires devront être pris en compte.
    \begin{itemize}
        \item L'administrateur sera le gestionnaire en chef
    de l'application. Il devra être en mesure de tout assurer
    au niveau de l'application. De ce fait, certaines tâches
    lui seront entièrement réservées.
        \begin{itemize}
            \item Ajouter/Afficher/Modifier/Supprimer des 
            utilisateurs à tous les niveaux.
            \item Ajouter/Afficher/Modifier/Supprimer des 
            cas exceptionnels dans le fonctionnement de la 
            bibliothèque.
            \item Faire tout ce que peut faire un autre utilisateur.
        \end{itemize}
        \item Le bibliothécaire sera le gestionnaire des 
    données accessibles de l’application. En d’autres termes, 
    il sera en interaction directe et continue avec les données 
    de la base. Ses tâches consisteront à:
        \begin{itemize}
            \item Ajouter/Afficher/Modifier/Supprimer un emprunt 
            ou une restitution  d’ouvrage
            \item Ajouter/Afficher/Modifier/Supprimer l’acquisition 
            de nouveaux ouvrages
        \end{itemize}
    \end{itemize}

    \item Abonnés
\\ Tous les abonnés hériteront des mêmes droits. Ce qui fera la 
différence sera le type qui leur sera attribué dès 
l’inscription. Trois catégories seront clairement 
identifiées.
    \begin{itemize}
    \item Un étudiant doit seulement s’acquitter d’une somme 
    forfaitaire pour une année afin d’avoir droit à tous les 
    services de la bibliothèque. Il sera identifié sous le type “inscrit”.
    \item L’accès à la bibliothèque est libre pour tous les enseignants.
    \item Il est possible d’autoriser des étudiants d’une autre université à 
    s’inscrire exceptionnellement comme abonné moyennant le versement d’une 
    cotisation. Il sera identifié sous le type “externe”.
    \end{itemize}
\end{enumerate}
\par 
    Les abonnés devront être en mesure de: \par 
\begin{enumerate}
    \item Consulter la disponibilité des ouvrages sans avoir à
     passer par les gestionnaires
    \item Vérifier la localisation des ouvrages à partir d’un catalogue
\end{enumerate}
    
\paragraph{Contraintes liées aux exigences des règlements de la bibliothèque}
Certaines règles sont strictes et exigibles afin d’assurer le bon fonctionnement 
de la bibliothèque. Par ailleurs, l’application devra prendre en compte toutes
 ces obligations pour que la cohérence soit respectée. \par
 \begin{enumerate}
     \item Le nombre d’abonnés externes est limité chaque année à environ 
     10\% des inscrits.
     \item Les ouvrages, souvent acquis en plusieurs exemplaires, sont 
     rangés dans des rayons de la bibliothèque. Chaque exemplaire est repéré 
     par une référence gérée dans le catalogue et le code du rayon où il est rangé.
     \item Chaque abonné ne peut emprunter plus de trois ouvrages.
     \item Le délai d’emprunt d’un ouvrage est de trois semaines; il peut 
     cependant être prolongé exceptionnellement à cinq(5) semaines.
 \end{enumerate}